\documentclass[a4paper]{article}

\usepackage[UTF8]{ctex}
\usepackage[a4paper,margin=1in]{geometry}
\usepackage{graphicx}
\usepackage{float}
\usepackage{listings}
\usepackage{longtable}
\usepackage{booktabs}
\usepackage{hyperref}
\usepackage{fancyhdr}
\usepackage{lastpage}
\usepackage{color}
\usepackage{indentfirst}        % 首段缩进
\setlength{\parindent}{2em}     % 缩进2字符
\usepackage{zhnumber}           % 中文编号
\usepackage[dvipsnames]{xcolor}
\usepackage{array}              % 表格增强
\usepackage{tabularx}           % 自适应表格
\usepackage{multirow}           % 多行合并

\newcommand{\college}{中山大学计算机学院}
\newcommand{\projname}{软件工程课程项目}
\newcommand{\reporttitle}{LifeMaster需求分析报告}
\newcommand{\stuno}{项目团队}
\newcommand{\authorname}{刘昊、彭怡萱、马福泉、林炜东、刘贤彬、刘明宇}
\newcommand{\major}{软件工程}
\newcommand{\adviser}{郑贵锋}
\newcommand{\startdate}{2025年3月1日}
\newcommand{\labenddate}{2025年7月6日}
\newcommand{\labroom}{计算机学院}

\pagestyle{fancy} % 使用 fancyhdr 风格
\fancyhf{}      % 清空默认的页眉页脚

% 设置页眉
\fancyhead[L]{\kaishu \projname}      % 左侧页眉显示项目名称
\fancyhead[C]{\kaishu \reporttitle}    % 中间页眉显示报告标题
\fancyhead[R]{\kaishu 项目团队} % 右侧页眉显示项目团队

% 设置页脚
\fancyfoot[C]{第 \thepage 页,共 \pageref{LastPage} 页} % 中间页脚显示页码

% 去除页眉页脚与正文之间的分隔线
\renewcommand{\headrulewidth}{0.4pt}
\renewcommand{\footrulewidth}{0pt}


\begin{document}

% 封面
\begin{titlepage}
    \centering
    

    \includegraphics[width=12cm]{img/SYSULogo.png}

    \vspace{1em}
    {\Large \college \par}
    \vspace{1em}
    {\Large \kaishu \projname \par}
    \vspace{3em}

    

      {\fontsize{40pt}{42pt}\kaishu \selectfont \boldmath \reporttitle\par}
    \vspace*{\fill}

    \begin{center}
    {\Large
    \makebox[5em][s]{项目名称}:\underline{\makebox[15em][c]{\kaishu LifeMaster}}\\[1em]
    \makebox[5em][s]{组员姓名}:\underline{\makebox[15em][c]{\kaishu 刘昊、彭怡萱、马福泉}}\\[0.5em]
    \makebox[5em][s]{}:\underline{\makebox[15em][c]{\kaishu 林炜东、刘贤彬、刘明宇}}\\[1em]
    \makebox[5em][s]{课程}:\underline{\makebox[15em][c]{\kaishu \major}}\\[1em]
    \makebox[5em][s]{课程教师}:\underline{\makebox[15em][c]{\kaishu \adviser}}\\[1em]
    \makebox[5em][s]{起始日期}:\underline{\makebox[15em][c]{\kaishu \startdate}}\\[1em]
    \makebox[5em][s]{结束日期}:\underline{\makebox[15em][c]{\kaishu \labenddate}}\\[1em]
    \makebox[5em][s]{学院}:\underline{\makebox[15em][c]{\kaishu \labroom}}
    }
    \end{center}

    \vspace*{\fill}
\end{titlepage}

% 目录
\tableofcontents
\newpage

\section{项目目标与愿景}

\subsection{项目愿景}

成为用户日常生活中最信赖的数字化生活管理伙伴,通过创新的功能设计和优质的用户体验,帮助用户实现高效的生活管理,提升生活品质,促进个人成长和社交互动。

\subsection{项目目标}

\begin{itemize}
    \item \textbf{功能完整性}:实现核心功能模块(手账、任务管理、财务管理),确保基本功能稳定可用
    \item \textbf{用户体验}:提供直观、易用的用户界面,新用户能在5分钟内完成基本功能操作
    \item \textbf{数据安全}:建立完善的用户数据保护机制,确保用户隐私和数据安全
    \item \textbf{性能指标}:支持1000并发用户,页面响应时间控制在3秒以内
\end{itemize}


\section{用户群体分析}

\subsection{核心用户群体}

\paragraph{学生群体(占比35\%)}
\begin{itemize}
    \item \textbf{特征}:年龄16-25岁,具备较强的数字化产品接受能力
    \item \textbf{需求}:学习笔记整理、错题管理、小组协作、生活记录
    \item \textbf{使用场景}:课堂笔记记录、复习资料整理、小组作业协作、校园生活分享
    \item \textbf{价值主张}:提高学习效率,丰富校园生活体验
\end{itemize}

\paragraph{职场人士(占比40\%)}
\begin{itemize}
    \item \textbf{特征}:年龄25-40岁,工作繁忙,注重效率和实用性
    \item \textbf{需求}:任务管理、时间规划、财务记录、工作心得记录
    \item \textbf{使用场景}:项目任务跟踪、会议记录整理、个人财务管理、工作经验总结
    \item \textbf{价值主张}:提升工作效率,实现工作生活平衡
\end{itemize}

\subsection{次要用户群体}

\paragraph{自由职业者(占比15\%)}
\begin{itemize}
    \item \textbf{特征}:时间相对自由,注重个人时间和财务管理
    \item \textbf{需求}:项目管理、收入支出跟踪、客户关系管理、行业交流
    \item \textbf{使用场景}:项目进度管理、财务状况分析、行业动态分享
    \item \textbf{价值主张}:优化工作流程,增强财务控制能力
\end{itemize}

\paragraph{生活爱好者(占比10\%)}
\begin{itemize}
    \item \textbf{特征}:热爱生活,喜欢记录和分享,注重美感和个性化
    \item \textbf{需求}:生活记录、美图分享、兴趣交流、个性化定制
    \item \textbf{使用场景}:日常生活记录、美食旅行分享、兴趣小组交流
    \item \textbf{价值主张}:丰富生活体验,扩展社交圈子
\end{itemize}

\section{用户需求分析}

\subsection{用户痛点识别}

\subsubsection{当前问题分析}

\begin{table}[H]
\centering
\begin{tabular}{|l|l|l|l|}
\hline
\textbf{问题类别} & \textbf{具体问题} & \textbf{影响程度} & \textbf{解决紧迫性} \\
\hline
\multirow{3}{*}{功能分散} & 需要使用多个不同应用 & 高 & 高 \\
& 数据无法统一管理 & 高 & 高 \\
& 操作流程繁琐复杂 & 中 & 中 \\
\hline
\multirow{3}{*}{协作困难} & 学习资料共享不便 & 高 & 高 \\
& 团队协作效率低下 & 中 & 中 \\
& 缺乏有效沟通渠道 & 中 & 低 \\
\hline
\multirow{3}{*}{体验不佳} & 界面设计不够美观 & 低 & 低 \\
& 个性化程度不足 & 中 & 中 \\
& 功能学习成本高 & 中 & 中 \\
\hline
\end{tabular}
\caption{用户痛点分析表}
\end{table}

\subsection{需求优先级分析}

\subsubsection{MoSCoW分析法}

\paragraph{Must Have(必须有)}
\begin{itemize}
    \item 用户注册登录功能
    \item 基础手账创建和编辑功能
    \item 任务的增删改查功能
    \item 收支记录的基本功能
    \item 数据持久化存储
    \item 基本的用户数据安全保护
\end{itemize}

\paragraph{Should Have(应该有)}
\begin{itemize}
    \item 手账的富文本编辑功能
    \item 任务提醒和分类功能
    \item 财务统计和可视化报表
    \item 小组创建和基础协作功能
    \item 响应式界面设计
    \item 数据导入导出功能
\end{itemize}

\paragraph{Could Have(可以有)}
\begin{itemize}
    \item 社区分享和互动功能
    \item 番茄钟专注功能
    \item 高级数据分析功能
    \item 个性化主题设置
    \item 第三方应用集成
    \item 移动端应用开发
\end{itemize}

\paragraph{Won't Have(暂不包含)}
\begin{itemize}
    \item 实时视频通话功能
    \item 复杂的企业级权限管理
    \item 多语言国际化支持
    \item 区块链技术集成
    \item 虚拟现实(VR)功能
\end{itemize}

\section{功能性需求}

\subsection{核心功能模块}

\subsubsection{用户管理模块}

\paragraph{用户注册}
\begin{itemize}
    \item \textbf{功能描述}:新用户可通过邮箱或手机号进行账户注册
    \item \textbf{输入条件}:有效的邮箱地址或手机号、符合安全要求的密码、基本个人信息
    \item \textbf{输出结果}:成功创建用户账户,自动登录并跳转到主界面
    \item \textbf{异常处理}:邮箱已存在、验证码错误、网络异常等情况的错误提示
\end{itemize}

\paragraph{用户登录}
\begin{itemize}
    \item \textbf{功能描述}:已注册用户通过凭据验证进入系统
    \item \textbf{输入条件}:注册邮箱/用户名、密码
    \item \textbf{处理过程}:验证用户凭据→检查账户状态→生成会话令牌→记录登录日志
    \item \textbf{输出结果}:登录成功跳转到用户主页,失败显示相应错误信息
\end{itemize}


\subsubsection{手账管理模块}

\paragraph{手账创建}
\begin{itemize}
    \item \textbf{功能描述}:用户可创建新的手账条目,支持富文本编辑
    \item \textbf{编辑功能}:
    \begin{itemize}
        \item 文本格式:字体、字号、颜色、加粗、斜体、下划线、删除线
        \item 段落格式:对齐方式、行间距、段落间距、项目符号、编号列表
        \item 插入元素:图片、链接、表格、分隔线
        \item 特殊功能:代码块、引用块、数学公式(可选)
    \end{itemize}
    \item \textbf{媒体支持}:支持jpg、png、gif格式图片,单张图片不超过100MB
    \item \textbf{自动保存}:每30秒自动保存草稿,防止数据丢失
\end{itemize}

\paragraph{手账分类管理}
\begin{itemize}
    \item \textbf{分类创建}:用户可自定义分类,设置分类名称、图标、颜色
    \item \textbf{标签系统}:支持为手账添加多个标签,实现细粒度分类
    \item \textbf{搜索功能}:支持按标题、内容、标签、日期等条件搜索手账
\end{itemize}

\subsubsection{任务管理模块}

\paragraph{任务基础操作}
\begin{itemize}
    \item \textbf{任务创建}:
    \begin{itemize}
        \item 基本信息:任务标题(必填)、详细描述、截止日期、优先级
        \item 分类标签:工作、学习、生活、健康等预设分类,支持自定义
        \item 关联功能:可关联相关手账、文件、联系人
    \end{itemize}
    \item \textbf{任务编辑}:支持修改所有任务属性,记录修改历史
    \item \textbf{任务删除}:软删除机制,支持30天内恢复,过期自动清理
\end{itemize}

\paragraph{任务状态管理}
\begin{itemize}
    \item \textbf{状态类型}:
    \begin{itemize}
        \item 待办(Todo):新创建的任务默认状态
        \item 进行中(In Progress):正在执行的任务
        \item 已完成(Done):已经完成的任务
        \item 已取消(Cancelled):因各种原因取消的任务
    \end{itemize}
    \item \textbf{状态转换}:支持拖拽方式快速改变任务状态
\end{itemize}

\paragraph{提醒与通知}
\begin{itemize}
    \item \textbf{提醒设置}:
    \begin{itemize}
        \item 时间提醒:截止前1天、1小时、30分钟等预设时间
        \item 自定义提醒:用户可设定具体的提醒时间
        \item 重复提醒:支持每日、每周、每月等重复模式
    \end{itemize}
    \item \textbf{通知方式}:浏览器通知、邮件通知(可选)
    \item \textbf{智能提醒}:基于用户习惯,智能推荐最佳提醒时间
\end{itemize}

\subsubsection{财务管理模块}

\paragraph{收支记录}
\begin{itemize}
    \item \textbf{快速记账}:
    \begin{itemize}
        \item 基本信息:金额(必填)、类型(收入/支出)、分类、日期
        \item 详细信息:备注说明、付款方式、商家信息、地点标记
        \item 凭证上传:支持上传小票、发票等凭证图片
    \end{itemize}
    \item \textbf{分类管理}:
    \begin{itemize}
        \item 预设分类:餐饮、交通、购物、娱乐、医疗、教育等
        \item 自定义分类:用户可创建个性化分类,设置图标和颜色
        \item 层级分类:支持二级分类,如餐饮→早餐/午餐/晚餐
    \end{itemize}
    \item \textbf{批量操作}:支持批量导入、批量修改、批量删除记录
\end{itemize}

\paragraph{统计分析}
\begin{itemize}
    \item \textbf{基础统计}:
    \begin{itemize}
        \item 时间维度:日、周、月、年度收支统计
        \item 分类维度:各类别收支占比、趋势分析
        \item 对比分析:同比、环比增长率计算
    \end{itemize}
    \item \textbf{可视化图表}:
    \begin{itemize}
        \item 饼状图:支出结构分析
        \item 柱状图:月度收支对比
        \item 折线图:收支趋势变化
        \item 热力图:消费习惯时间分布
    \end{itemize}
    \item \textbf{智能洞察(后续计划)}:
    \begin{itemize}
        \item 消费模式识别:发现用户消费规律和异常
        \item 节约建议:基于数据分析提供理财建议
        \item 目标跟踪:预算执行情况监控和预警
    \end{itemize}
\end{itemize}

\paragraph{预算管理}
\begin{itemize}
    \item \textbf{预算设置(后续计划)}:
    \begin{itemize}
        \item 总体预算:月度、季度、年度总支出预算
        \item 分类预算:各支出类别的详细预算分配
        \item 弹性预算:设置预算的浮动范围
    \end{itemize}
    \item \textbf{预算监控}:
    \begin{itemize}
        \item 实时跟踪:预算使用百分比实时显示
        \item 预警机制:达到80\%、90\%、100\%时分级预警
        \item 超支分析:超支原因分析和改进建议
    \end{itemize}
\end{itemize}

\subsection{扩展功能模块}

\subsubsection{小组协作模块}

\paragraph{小组管理}
\begin{itemize}
    \item \textbf{小组创建}:用户可创建学习小组、工作团队等协作组织
    \item \textbf{成员管理}:邀请成员、设置权限级别、移除成员
    \item \textbf{角色权限}:组长、管理员、普通成员等不同权限级别
\end{itemize}

\paragraph{内容共享}
\begin{itemize}
    \item \textbf{资料共享}:手账、学习资料、任务清单的小组内共享
    \item \textbf{协同编辑}:支持多人同时编辑共享文档(基础版本)
    \item \textbf{版本管理}:记录文档修改历史,支持版本回滚
\end{itemize}

\subsubsection{社区分享模块(后续计划)}

\paragraph{内容发布}
\begin{itemize}
    \item \textbf{发布类型}:学习经验、生活技巧、手账展示、资源分享
    \item \textbf{内容审核}:基础的内容审核机制,防止不当内容
    \item \textbf{隐私控制}:公开、仅好友可见、私密等隐私级别
\end{itemize}

\paragraph{互动功能}
\begin{itemize}
    \item \textbf{基础互动}:点赞、评论、转发、收藏
    \item \textbf{关注机制}:关注其他用户,接收动态更新
    \item \textbf{话题标签}:通过话题标签发现相关内容
\end{itemize}

\subsubsection{番茄钟模块(后续计划)}

\paragraph{专注功能}
\begin{itemize}
    \item \textbf{时间设置}:自定义专注时长(默认25分钟)和休息时长(默认5分钟)
    \item \textbf{干扰屏蔽}:专注期间屏蔽非紧急通知
    \item \textbf{统计记录}:记录每日专注时长和效率数据
\end{itemize}

\section{非功能性需求}

\subsection{性能需求}

\subsubsection{响应时间要求}

\begin{table}[H]
\centering
\begin{tabular}{|l|l|l|}
\hline
\textbf{操作类型} & \textbf{期望响应时间} & \textbf{最大可接受时间} \\
\hline
页面加载 & $\leq$ 2秒 & $\leq$ 3秒 \\
\hline
用户登录 & $\leq$ 1秒 & $\leq$ 2秒 \\
\hline
数据查询 & $\leq$ 0.5秒 & $\leq$ 1秒 \\
\hline
文件上传(5MB) & $\leq$ 10秒 & $\leq$ 15秒 \\
\hline
数据保存 & $\leq$ 0.3秒 & $\leq$ 0.5秒 \\
\hline
报表生成 & $\leq$ 3秒 & $\leq$ 5秒 \\
\hline
\end{tabular}
\caption{系统响应时间要求}
\end{table}

\subsubsection{并发性能要求}

\begin{itemize}
    \item \textbf{并发用户数}:系统应支持至少100个并发用户同时在线
    \item \textbf{数据库性能}:单个数据库查询响应时间不超过100ms
    \item \textbf{文件处理}:支持同时处理至少10个文件上传请求
\end{itemize}

\subsubsection{吞吐量要求}

\begin{itemize}
    \item \textbf{事务处理}:每秒至少处理50个业务事务
    \item \textbf{数据处理}:每小时能处理至少1GB的用户数据
    \item \textbf{备份恢复}:数据库备份时间不影响正常业务操作
\end{itemize}

\subsection{可靠性需求}

\subsubsection{数据备份}

\begin{itemize}
    \item \textbf{备份频率}:核心数据每日备份,重要配置每周备份
    \item \textbf{备份保留}:数据备份保留期不少于30天
    \item \textbf{异地备份}:实现异地数据备份
\end{itemize}

\subsection{安全性需求}

\subsubsection{身份认证}

\begin{itemize}
    \item \textbf{密码策略}:
    \begin{itemize}
        \item 最小长度8位,包含字母、数字、特殊字符
        \item 支持密码强度实时检测和建议
    \end{itemize}
    \item \textbf{会话管理}:
    \begin{itemize}
        \item 会话超时时间为2小时
        \item 支持强制登出功能
        \item 异常登录行为检测和通知
    \end{itemize}
    \item \textbf{多重验证}:重要操作支持邮箱验证码二次确认
\end{itemize}

\subsubsection{数据保护(后续计划)}

\begin{itemize}
    \item \textbf{传输加密}:所有数据传输使用HTTPS/TLS 1.2以上协议
    \item \textbf{存储加密}:敏感数据(密码、个人信息)采用AES-256加密存储
    \item \textbf{访问控制}:实现基于角色的访问控制(RBAC)
\end{itemize}

\subsubsection{隐私保护}

\begin{itemize}
    \item \textbf{数据收集}:明确告知用户数据收集目的和范围
    \item \textbf{用户控制}:用户可查看、修改、删除自己的个人数据
    \item \textbf{第三方集成}:与第三方服务集成时保护用户隐私
    \item \textbf{合规要求}:遵守相关数据保护法律法规
\end{itemize}

\subsection{可用性需求}

\subsubsection{用户界面}

\begin{itemize}
    \item \textbf{易学性}:新用户能在5分钟内掌握基本操作
    \item \textbf{易用性}:常用功能不超过3次点击可达
    \item \textbf{一致性}:界面设计风格统一,操作逻辑一致
    \item \textbf{反馈机制}:操作结果有明确的视觉反馈
\end{itemize}

\subsubsection{可访问性}

\begin{itemize}
    \item \textbf{浏览器兼容}:支持Chrome 90+、Firefox 88+、Safari 14+、Edge 90+
    \item \textbf{设备适配}:支持1920×1080及以上分辨率的桌面设备
    \item \textbf{网络环境}:在2Mbps带宽下能正常使用基本功能
    \item \textbf{无障碍访问}:支持键盘导航,色彩对比度符合WCAG 2.1标准
\end{itemize}

\subsubsection{国际化支持}

\begin{itemize}
    \item \textbf{语言支持}:当前版本支持简体中文
    \item \textbf{字符编码}:全面支持UTF-8编码
    \item \textbf{时区处理}:正确处理不同时区的时间显示
    \item \textbf{扩展性}:界面设计考虑未来多语言扩展需求
\end{itemize}

\subsection{可维护性需求}

\subsubsection{代码质量}

\begin{itemize}
    \item \textbf{代码规范}:遵循统一的编码规范和命名约定
    \item \textbf{注释文档}:关键模块和复杂逻辑有充分的注释说明
    \item \textbf{模块化设计}:采用模块化架构,降低耦合度
    \item \textbf{代码复用}:公共功能组件化,提高代码复用率
\end{itemize}

\subsubsection{测试要求}

\begin{itemize}
    \item \textbf{单元测试}:核心业务逻辑单元测试覆盖率不低于80\%
    \item \textbf{集成测试}:关键业务流程有完整的集成测试用例
    \item \textbf{性能测试}:定期进行性能测试,确保系统性能不退化
    \item \textbf{安全测试}:定期进行安全漏洞扫描和渗透测试
\end{itemize}

\subsubsection{运维支持}

\begin{itemize}
    \item \textbf{日志系统}:完善的日志记录机制,支持问题追踪和性能分析
    \item \textbf{监控告警}:系统关键指标监控和异常告警机制
    \item \textbf{配置管理}:支持动态配置更新,无需重启服务
    \item \textbf{版本管理}:规范的版本发布和回滚机制
\end{itemize}

\section{系统约束条件}

\subsection{技术约束}

\subsubsection{开发环境约束}

\begin{itemize}
    \item \textbf{编程语言}:后端使用Python 3.8+,前端使用HTML5/CSS3/JavaScript
    \item \textbf{框架选择}:后端框架限定为Flask,前端使用原生技术栈
    \item \textbf{数据库}:使用MySQL 5.7+作为主数据库
    \item \textbf{开发工具}:统一使用Git进行版本控制,GitHub作为代码托管平台
\end{itemize}

\subsubsection{部署环境约束}

\begin{itemize}
    \item \textbf{操作系统}:支持Linux(Ubuntu 18.04+)和Windows Server部署
    \item \textbf{Web服务器}:支持Apache或Nginx作为反向代理
    \item \textbf{硬件要求}:最低4GB内存,推荐8GB以上;最低2核CPU,推荐4核以上
    \item \textbf{存储要求}:系统盘不少于100GB,数据盘根据用户规模扩展
\end{itemize}

\subsection{业务约束}

\subsubsection{功能范围约束}

\begin{itemize}
    \item \textbf{用户规模}:初期版本设计支持用户规模不超过10万人
    \item \textbf{数据存储}:单用户数据存储限制为1GB
    \item \textbf{文件大小}:单个上传文件不超过50MB
    \item \textbf{小组规模}:单个小组成员数量不超过100人
\end{itemize}

\subsubsection{合规约束}

\begin{itemize}
    \item \textbf{数据保护}:遵守《个人信息保护法》等相关法律法规
    \item \textbf{内容审核}:建立内容审核机制,确保平台内容合规
    \item \textbf{知识产权}:尊重知识产权,建立版权保护机制
    \item \textbf{服务条款}:制定明确的用户服务协议和隐私政策
\end{itemize}

\section{总结}

\subsection{需求分析概要}

本需求分析报告全面梳理了LifeMaster生活管理应用的核心需求,明确了项目的发展目标和功能边界。通过深入的用户研究和需求分析,我们确定了以手账记录、任务管理、财务管理为核心的功能架构,并规划了小组协作和社区分享等扩展功能。

\subsection{核心价值定位}

LifeMaster致力于为用户提供一站式的生活管理解决方案,解决现有应用功能分散、操作繁琐、协作困难等痛点。项目的核心价值在于:

\begin{itemize}
    \item \textbf{功能整合}:将多种生活管理功能集成在一个平台,提升用户使用效率
    \item \textbf{用户体验优先}:注重界面设计和交互体验,降低用户学习成本
    \item \textbf{数据安全}:建立完善的数据保护机制,保障用户隐私和数据安全
\end{itemize}

\subsection{关键成功要素}

项目成功的关键要素包括:

\begin{itemize}
    \item \textbf{技术架构稳定}:采用成熟可靠的技术栈,确保系统稳定性和可扩展性
    \item \textbf{需求理解准确}:深入理解用户需求,确保产品功能与用户期望匹配
    \item \textbf{开发进度可控}:合理规划开发周期,确保项目按时交付
    \item \textbf{质量保证机制}:建立完善的测试和质量保证流程,确保产品质量
\end{itemize}

\subsection{后续工作方向}

基于本需求分析报告,项目后续工作重点包括:

\begin{enumerate}
    \item \textbf{系统架构设计}:根据功能需求和性能要求,设计合理的系统架构
    \item \textbf{数据库设计}:基于数据需求,设计高效稳定的数据库结构
    \item \textbf{界面原型设计}:根据用户体验需求,设计直观易用的用户界面
    \item \textbf{开发实施计划}:制定详细的开发计划和里程碑,确保项目有序推进
    \item \textbf{测试策略制定}:基于质量要求,制定全面的测试策略和验收标准
\end{enumerate}

本需求分析报告将作为项目开发的重要指导文档,确保团队对项目目标和要求有清晰统一的认识,为项目的成功实施奠定坚实基础。

\end{document}

\subsection{功能验收标准}

\subsubsection{核心功能验收}

\begin{table}[H]
\centering
\begin{tabular}{|l|l|l|}
\hline
\textbf{功能模块} & \textbf{验收标准} & \textbf{测试方法} \\
\hline
\multirow{3}{*}{用户管理} & 用户注册成功率$\geq$95\% & 100个测试用例注册 \\
& 登录响应时间$\leq$2秒 & 压力测试验证 \\
& 密码安全策略正确执行 & 安全测试验证 \\
\hline
\multirow{3}{*}{手账管理} & 支持所有规定的富文本格式 & 功能测试验证 \\
& 图片上传成功率$\geq$98\% & 批量上传测试 \\
& 自动保存功能正常工作 & 异常中断测试 \\
\hline
\multirow{3}{*}{任务管理} & 任务CRUD操作100\%成功 & 完整业务流程测试 \\
& 提醒通知准确及时 & 时间精度测试 \\
& 状态转换逻辑正确 & 状态机测试 \\
\hline
\multirow{3}{*}{财务管理} & 金额计算精度100\%准确 & 精度测试验证 \\
& 报表生成时间$\leq$5秒 & 性能测试验证 \\
& 数据导入导出功能正常 & 兼容性测试 \\
\hline
\end{tabular}
\caption{核心功能验收标准}
\end{table}

\subsection{性能验收标准}

\subsubsection{响应时间验收}

\begin{itemize}
    \item \textbf{页面加载}:90\%的页面加载时间在3秒以内
    \item \textbf{数据查询}:95\%的查询操作在1秒内完成
    \item \textbf{文件上传}:5MB文件上传在15秒内完成
    \item \textbf{报表生成}:财务报表生成在5秒内完成
\end{itemize}

\subsubsection{并发性能验收}

\begin{itemize}
    \item \textbf{并发用户}:支持1000并发用户同时在线,系统响应正常
    \item \textbf{数据库连接}:数据库连接池配置合理,无连接泄漏
    \item \textbf{内存使用}:系统内存使用率在高负载下不超过80\%
    \item \textbf{CPU使用}:CPU使用率在正常负载下不超过70\%
\end{itemize}

\subsection{安全验收标准}

\subsubsection{安全功能验收}

\begin{itemize}
    \item \textbf{身份认证}:通过安全认证测试,无绕过漏洞
    \item \textbf{数据加密}:敏感数据传输和存储加密有效
    \item \textbf{权限控制}:用户权限隔离测试100\%通过
    \item \textbf{SQL注入}:通过SQL注入攻击测试,无安全漏洞
\end{itemize}

\subsection{用户体验验收标准}

\subsubsection{易用性验收}

\begin{itemize}
    \item \textbf{学习成本}:新用户5分钟内能完成基本操作
    \item \textbf{操作效率}:常用功能3次点击内可达成
    \item \textbf{错误处理}:用户操作错误有明确提示和恢复指导
    \item \textbf{界面一致性}:界面设计风格统一,用户体验连贯
\end{itemize}

\section{风险分析}

\subsection{技术风险}

\subsubsection{开发技术风险}

\begin{table}[H]
\centering
\begin{tabular}{|l|l|l|l|}
\hline
\textbf{风险项} & \textbf{概率} & \textbf{影响} & \textbf{应对措施} \\
\hline
技术栈学习曲线陡峭 & 中 & 高 & 提前技术培训,设置学习缓冲期 \\
\hline
数据库性能瓶颈 & 中 & 中 & 数据库优化,索引设计,缓存策略 \\
\hline
前后端集成困难 & 低 & 中 & 及早接口设计,增量集成测试 \\
\hline
第三方依赖风险 & 低 & 高 & 选择稳定的开源组件,准备备选方案 \\
\hline
\end{tabular}
\caption{技术风险分析}
\end{table}

\subsection{项目风险}

\subsubsection{进度风险}

\begin{itemize}
    \item \textbf{需求变更风险}:需求理解偏差导致返工
    \item \textbf{人员变动风险}:关键开发人员离开影响进度
    \item \textbf{技术难度风险}:技术实现复杂度超出预期
    \item \textbf{测试风险}:测试发现重大缺陷需要大量修复时间
\end{itemize}

\subsection{业务风险}

\subsubsection{用户接受度风险}

\begin{itemize}
    \item \textbf{市场竞争}:同类产品竞争激烈,用户选择多样
    \item \textbf{用户习惯}:用户现有使用习惯难以改变
    \item \textbf{功能匹配度}:产品功能与用户实际需求不匹配
    \item \textbf{学习成本}:产品使用门槛过高影响用户接受度
\end{itemize}

\section{总结}

\subsection{需求分析总结}

本需求分析报告通过深入的用户研究和市场分析,明确了LifeMaster生活管理应用的核心价值定位和功能范围。项目旨在为用户提供一站式的生活管理解决方案,涵盖手账记录、任务管理、财务管理等核心功能,并通过小组协作和社区分享增强用户粘性。

\end{document}
